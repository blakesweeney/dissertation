This disssertation contians two types of work. The first, as described in
Chapter 2 is the creation and maintaince of our data pipeline. This chapter
focuses on the techinqual work behind the extension of our pipeline. In general,
this work extends our previous pipeline to import more data as well as
standarizing several parts of the pipeline. As a result, my work has provided a
framework for future modifications of the pipeline. This work was driven both by
the move from RNA 3D structures being provieded in mmCIF instead of the outdated
PDB format, as well as, the need to cleanup the previous version of the
pipeline.

The second type of work is scientific. In Chapters 3-5 I discuss my work on
creating equivelance sets for all RNA 3D structures (Chapter 3), then discuss
using these sets to build representative sets (Chapter 4) and then how to use
these representative sets along with new quality data to select a set of high
quality loops for future analysis (Chapter 5). The work in Chapters 3 and 4 was
driven by the move from PDB to mmCIF formats. This move forced the redesign of
the previous method, as it would only use the largest chain in file. This change
allowed me reconsider the approach and allowed several improvements. 

The work in Chapter 5 was prompted by the release of new structure quality data,
Real Space R Z-Score (RSRZ). This data allows the examination of how well a
proposed structure fits the data it is built from. By using this we can limit
our studies of RNA loops to only those that are from high quality well modeled
structures. 
