This dissertation contains two types of work. The first is the creation and
maintenance of our data pipeline. This chapter focuses on the technical work
behind the extension of our pipeline. In general, this work extends our
previous pipeline to import more data as well as standardizing several parts of
the pipeline. As a result, this work provides a framework for future
modifications of the pipeline. This work was driven both by the move from RNA
3D structures being provided in mmCIF format instead of the more limited PDB
format, as well as the need to clean up the previous version of the pipeline.

The second type of work is scientific including my work on creating equivalence
classes for all RNA 3D structures, using these sets to build representative
sets and then how to use these representative sets along with new quality data
to select a set of high quality loops for future analysis. The new work on
equivalence classes and representative sets was driven by the move from PDB to
mmCIF formats. This move forced the redesign of the previous method, as it
would only use the largest chain in each PDB file. This change allowed me
to reconsider the approach and allowed several improvements. 

The work on loop quality was prompted by the release of new structure quality
data, Real Space R Z-Score (RSRZ). This data allows the examination of how well
a proposed structure fits the data it is built from. By using this we can limit
our studies of RNA loops to only those that are from high quality, well modeled
structures. 

% vim ft=markdown
