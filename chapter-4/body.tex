\chapter{SELECTION OF THE REPRESENTATIVE SETS FROM THE EQUIVALENCE CLASSES OF
RNA CONTAINING 3D STRUCTURES}

\section{Introduction and motivation}

The previous chapter discussed a method for organizing all RNA-containing 3D
structures in PDB into sets of equivalence classes. This procedure structures
the 3D dataset. However, for many use cases, simply organizing the data is
insufficient. In particular studies exploring the statistical properties of the
structural data set will be biased toward 3D structures that are highly
represented, as is currently the case with tRNA and ribosome structures. A
summary of is shown in Table~\ref{tab:mol-dist}.

\begin{table}
  \begin{tabulary}{\linewidth}{LRR}
    \toprule
    Molecule &
      Number of Integrated Functional Elements in the class &
      Percent of all RNA containing structures in PDB as of Sept 09, 2016 \\
    \midrule
    \TT{} LSU            & 282  & 4.0 \\
    \TT{} SSU            & 379  & 5.4 \\
    \TT{} 5S             & 282  & 4.0 \\
    \EC{} LSU            & 161  & 2.2 \\
    \EC{} SSU            & 157  & 2.2 \\
    \EC{} 5S             & 156  & 2.2 \\
    \HM{} LSU            & 69   & 0.9 \\
    \HM{} 5S             & 67   & 0.9 \\
    \DR{} LSU            & 42   & 0.5 \\
    \SC{} SSU            & 56   & 0.8 \\
    \SC{} LSU            & 63   & 0.9 \\
    Other Ribosomes      &      & \\
    Ribosomal Subtotal   & 1714 & 24.5 \\
    Remaining Structures & 5287 & 75.5 \\
    Total                & 7001 & 100 \\
    \bottomrule
  \end{tabulary}
  \caption{Proportion of solved structures that are from bacterial and yeast
    ribosomes. This table shows data from the 2.92 release of NR set at the
    'all' resolution, availabe at:
    \url{http://rna.bgsu.edu/rna3dhub/nrlist/release/2.92/all}. This dataset contains
    all structures available as of Sept 09, 2016. This table presents the
    fraction of the total structural database that comprises structures from all
    sources ribosomes. In total they make up 20\% of the solved crystal
  structures. LSU: Large Ribosomal Subunit, SSU: Small Ribosomal Subunit.}
  \label{tab:mol-dist}
\end{table}

As seen in the table, \EC{} and T. thermophilus ribosomes alone make up 20% of
the entire dataset and all ribosomal total NNN. For many statistical analyses of
RNA structure, it is desirable to identify one high quality representative
structure \cite{Leontis2012b}. For example, such a reduced redundancy dataset
set is desirable for constructing the RNA 3D Motif Atlas \cite{Petrov2013}. In
previous work a method to identify representative structures  from PDB format
files, called the non-redundant (NR) set was developed. We have updated the
method to take advantage of the new mmCIF data and fix limitations of the
previous method. These were posted on a weekly basis from February 5, 2011 to
December 5, 2014 on the BGSU RNA site and made available for searching on NDB
(\url{http://ndbserver.rutgers.edu/}).

\section{Requirements for representative set}

The set of representative structures has a variety of uses in structural
bioinformatics applications. Most notably it supports population and regular
updating of the RNA 3D Motif Atlas \cite{http://rna.bgsu.edu/rna3dhub/motifs}.
The RNA 3D Motif Atlas is a clustering of all high quality loops extracted from
all structures in the current Representative Set of structures. For this reason
it is desirable to avoid frequent changes in the representative structure due to
insignificant increments in the selection criteria. If we allow frequent
insignificant changes in the representative in structures selected for the
Representative set, these changes cause large changes in the RNA 3D Motif Atlas.
This makes the resource appear more unstable than it actually is and so we aim
to avoid changes due to minor improvements in structures. Thus the goals are of
the improvements described in this chapter are:

\begin{enumerate}
  \item The representative set should reflect observed variation in sequence,
    structure, and biological variation in the set of solved 3D structures

  \item The representative set should contain highest quality representative
    X-Ray structures from each Equivalence Class. NMR or Cyro-EM structures are
    chosen as representatives only when no X-ray structure is available.

  \item The representative set should only change upon the appearance of new,
    significantly better structures, using criteria explained in this chapter.
\end{enumerate}

The representative set is intended to display the observed structural, sequence,
and biological variation present in the current RNA-containing structural
dataset, avoiding duplication as much as possible. Each unique structural class
should be represented by one high quality structure, as the classes already take
into account the possibility of large structural variation, as explained in the
previous chapter. It is desirable to pick the highest quality structure
available from each equivalence class. A high quality structure is one that
models the experimental data as well or better than other available structures
and is as complete as possible.

\section{Methodology for selecting representative RNA 3D structures from
Equivalence classes}

This dissertation describes new methodology that represents significant
improvements over the previously developed methodologies of the BGSU RNA
resource. In the previous method, the Representative Structure from each
Equivalence Class was selected based solely upon the number of annotated base
pairs per nucleotide, discussed in detail in \cite{Leontis2012b}.

As the goal is to use only high quality structures the procedure uses X-ray
structures as the representative, if possible. Some structures, such as that of
the the mouse ribosome (3J7Q), have only been solved using cryo-electron
microscopy (``cryo-EM'') and in these cases, and only these cases,
representative selected for the equivalence set is necessarily a cryo-EM
structure. Only X-Ray structures are used when available because methods for
assessing the modeling quality of cryo-EM structures, comparable to RSR-Z scores
for X-ray crystallography are still under development and are not yet available
available (Berman, personal communication). Having object assessment data, like
RSR-Z scores available is essential for downstream applications such as
selecting reliable loops for inclusion in the Motif Atlas. As soon as
reliability data comparable to RSRZ scores are available for cryo-EM structures,
these structures will be treated on an equivalent footing as X-ray.

The procedure for each possible structure in:

\begin{enumerate}
  \item Select the structure with the most number of annotated basepairs per
    nucleotide.

  \item Find a structures with at least 1\% more bases and base pairs and select
    it as the current representative

  \item Repeat step 2 until there no structure has more base pairs and bases.
\end{enumerate}

This procedure has a few features that are worth discussing in detail. First, we
use the number of FR3D annotated cis Watson Crick/Watson Crick base pairs per
nucleotide (BP/NT) as the measure of structural quality because in our
experience a high quality model of the same RNA molecule contains more annotated
cWW base pairs than a poor model. This is a useful proxy and not a direct
measure of the modeling quality such as that provided by Real Space Refinement
(RSR) which is provided by PDB for all structures. RSR is measures how well a
structural model fits the underlying experimental data and is discussed in more
detail in the next chapter.

Secondly, this method is designed to select a structure based both on BP/NT but
also the number of nucleotides. Structures in the same class vary in the number
of resolved nucleotides. If we use only BP/NT then we may select a  smaller
structure that omits parts of the 3D structure that are less well modeled. We
prefer more complete representative structures over smaller structures and thus
include this second criterion.

Finally, the new method requires new representatives to show a significant
increase in both the number of nucleotides and the BP/NT. By doing so, we avoid
changing representatives upon insignificant changes in overall quality. Thus
stabilizing the Representative Set overtime. This 1\% increase is an empirically
determined cutoff that may be adjusted in the future. The new method of
representative selection described in this dissertation will be referred to
henceforth as the ``dual 1\% method''.

To describe the overall logic of the logic I have added a figure of the
pseudocode in Figure~\ref{fig:pseudocode-representatives}. This figure shows the
overall logic of this procedure, but does not correspond exactly to the
implementation.

\begin{figure}
  \caption{Pseudocode showing the overall logic of the selection for
  representative sets.}
  \label{fig:pseudocode-representatives}
\end{figure}

\section{Results of building the representative set}

I evaluated the Representative Set selected as described above by using the new
dual 1\% increase method, and compared results with the previous method and a
modification of the current method that changes the representative structure
based upon improvement of just one criterion. These classes used all available
structures from July 28, 2016. I will first discuss the differences between
structures selected by these methods and then move on to to examine the
stability of the methods.

\subsection{Selection of Structures for the Representative Set}

I evaluated the selection of representatives by examining the selection of
representatives from the equivalence classes built in the previous chapter. This
used all structures available as of July 25, 2016 (equivalent to release 2.86).
I examined classes for changes and determined where the new method selected
different representatives than the other methods. These changes were then
examined manually to assess the quality of the structures using the original as
well as additional criteria. There were 29 classes in the data set for which
different representatives were selected by the two methods. Shown in
Figure~\ref{fig:hm-lsu-rep} is a summary of the representative selection for the
three methods discussed here. The upper panel shows the previous method where we
can see the structure that is at the top of stack of very similar structures is
selected as the representative. This contrasts with our 1% increase method which
selects a example with more nucleotides and basepairs but a worse ratio of the
two. This selection is our desired result.

\begin{figure}
  \caption{Scatter plot of representative selection across three possible
    methodologies. Shown here is a summary of which IFE’s are selected as the
    representative for the 3 method discussed here. Each panel indicates a
    different method, in order from top to bottom, the previous method, a 1%
    increase method and the any increase method. Cyro-em structures are shown as
    circles, while X-ray are shown as triangles.  The representative IFE is
    colored in blue, while the other members of the set are colored red. The
    upper panel shows the previous method. It is difficult to see but the IFE
    selected as the representative is the IFE at the top of the stack of IFE’s
  in the upper right.}
  \label{fig:hm-lsu-rep}
\end{figure}

We then evaluated the selection of representative in context of the resolution
distribution. The ideal selection has the highest resolution of any structure in
the class. Show in Figure~\ref{fig:hm-rep-res-dist} is a summary. In it we can
see that all three methods select an IFE with resolution 2.4{\AA}. This is
consistent with our goals for the methods.

\begin{figure}
  \caption{Distribution of resolutions for IFE’s in NR\_all\_65634.1 with the
    resolution of the representative indicated. This histogram shows the
    resolution distribution of all IFE’s in the class. The red line indicates
  the resolution of the representative chosen by all 3 methods.}
  \label{fig:hm-rep-res-dist}
\end{figure}

We further evaluated the method described here by examining the selection of
representatives for the \EC{} Large and Small Subunits as well as the Thermus
thermophilus Large and Small Ribosomal Subunits, shown in
Figure~\ref{fig:ec-lsu-rep}, Figure~\ref{fig:ec-ssu-rep},
Figure~\ref{fig:tt-lsu-rep} and Figure~\ref{fig:tt-ssu-rep} respectively. As we
can see in these figures, only the \EC{} LSU T. thermophilus SSU differ in
selection of representatives.

\begin{figure}
  \caption{Scatter plot of the number of base pairs vs the number of nucleotides
    for selected \EC{} Large Subunits using the three methods discussed here.
    The representative selected is shown as a large blue dot, while all other
    members are shown as small red dots. The plot has been truncated to show
    only the structures with at least 2840 nucleotides and 500 basepairs for
  clarity.}
  \label{fig:ec-lsu-rep}
\end{figure}

\begin{figure}
  \caption{Scatter plot of the number of base pairs vs the number of nucleotides
    for selected \EC{} Small Subunits using the three methods discussed here.
    The representative selected is shown as a large blue dot, while all other
    members are shown as small red dots. The plot has been truncated to show
    only the structures with at least 1400 nucleotides and 300 basepairs for
  clarity.}
  \label{fig:ec-ssu-rep}
\end{figure}

\begin{figure}
  \caption{Scatter plot of the number of base pairs vs the number of nucleotides
    for selected T. thermophilus Large Subunits using the three methods
    discussed here. The representative selected is shown as a large blue dot,
    while all other members are shown as small red dots. The plot has been
    truncated to show only the structures with at least 2725 nucleotides and 500
  basepairs for clarity.}
  \label{fig:tt-lsu-rep}
\end{figure}

\begin{figure}
  \caption{Scatter plot of the number of base pairs vs the number of nucleotides
    for selected T. thermophilus Small Subunits using the three methods
    discussed here. The representative selected is shown as a large blue dot,
    while all other members are shown as small red dots. The plot has been
    truncated to show only the structures with at least 1490 nucleotides and 200
  basepairs for clarity.}
  \label{fig:tt-ssu-rep}
\end{figure}

\subsubsection{Evaluating representative selection on the basis of resolution}

Another useful criteria for evaluation our representative selection is to
examine if our method selects the structure with lowest resolution. Generally,
higher resolution structures are better than lower resolution structures. To
explore this I computed the difference between the selected representative for
all groups and the structure in the group with the smallest resolution. A
summary of this data is shown in Table~\ref{tab:res-diff-summary}.

\begin{table}
  \begin{tabulary}{\linewidth}{LRRR}
    \toprule
    Method &  Representatives with lowest resolution &  Representatives with
    resolution within 1{\AA}  & Representatives with resolution > 1{\AA} \\
    \midrule
    Previous          & 1506 (93\%) & 86 (5.3\%) & 14 (0.8\%) \\
    Dual Increase     & 1494 (93\%) & 97 (6.0\%) & 15 (0.9\%) \\
    Dual 1\% Increase & 1506 (93\%) & 88 (5.4\%) & 12 (0.7\%) \\
    \bottomrule
  \end{tabulary}
  \caption{A summary of the differences between the resolution of the
  representatives as compared to the member of the group with the lowest
  resolution.}
  \label{tab:res-diff-summary}
\end{table}

From this we can see that nearly all groups use a structure with the lowest
resolution, as we desired. However, roughly 5\% of all groups have a
representative with resolution with 1{\AA} of the minimum. This leaves a very small
percent, less than 1\%, of all groups that use a representative with resolution
greater than 1{\AA} from the minimum. Shown in Figure~\ref{fig:res-diff-histogram}
is a histogram that displays the differences for all groups where the difference
is greater than 0{\AA}.

\begin{figure}
  \caption{Histogram of the differences between the resolution of the selected
    representative and the minimum resolution within the group for all 3
    methods. This histogram does not show the groups where the representative
    resolution is the same as the minimum resolution.  The red vertical line
    indicates the cutoff of 1{\AA}. The numbers to the left of the red box indicate
    the total number of groups with resolution difference less than 1{\AA}, while
    the number to the right indicates the number of groups with resolution
  difference greater than 1{\AA}.}
  \label{fig:res-diff-histogram}
\end{figure}

From this figure we can see two things. First, all methods are very similar in
terms of the number of groups that select representatives with unusually high
resolution. Secondly, several groups have very large differences between the
selected representative and the minimum resolution. For example one group has a
difference of 14\AA. Examining this data shows that there are 19 groups with
resolution differences. Shown in Table~\ref{rep-res-diff-details} are the
summary of those groups.

\begin{table}
  \begin{tabular}{llrrr}
    \toprule
    Group &  Minimum Resolution ({\AA}) &  Previous &  Dual Increase &  Dual 1\% Increase \\
    \midrule
    NR\_all\_00304.1 &  9.0  & \ife{3J0D}{1}{A} (11.1{\AA}) [2.1{\AA}]  &
                               \ife{3J0D}{1}{A} (11.1{\AA}) [2.1{\AA}]  &
                               \ife{3J0D}{1}{A} (11.1{\AA}) [2.1{\AA}]  \\
    NR\_all\_13601.2 &  5.0  & \ife{4KZZ}{1}{j} (7.03{\AA}) [2.03{\AA}] &
                               \ife{4KZZ}{1}{j$ (7.03{\AA}) [2.03{\AA}] \\
    NR\_all\_14586.1 &  3.5  & \ifePair{4V6X}{1}{A5}{4V6X}{1}{A8} (5{\AA}) [1.5{\AA}] &
                               \ifePair{4V6X}{1}{A5}{4V6X}{1}{A8} (5{\AA}) [1.5{\AA}] \\
    NR\_all\_16577.1 &  6.9  & \ife{3J0O}{1}{V} (9{\AA}) [2.1{\AA}] &
                               \ife{3J0O}{1}{V} (9{\AA}) [2.1{\AA}] &
                               \ife{3J0O}{1}{V} (9{\AA}) [2.1{\AA}] \\
    NR\_all\_18070.1 &  2.4  & \ife{4TUD}{1}{QV} (3.6{\AA}) [1.2{\AA}] \\
    NR\_all\_33599.2 &  2.6  & \ife{4V5M}{1}{AV} (7.8{\AA}) [5.2{\AA}] \\
    NR\_all\_33884.1 &  3.45 & \ife{4KZY}{1}{i} (7.01{\AA}) [3.56{\AA}] &
                               \ife{4KZY}{1}{i} (7.01{\AA}) [3.56{\AA}] \\
    NR\_all\_37074.2 &  2.1  & \ife{4V6Z}{1}{BB} (12{\AA}) [9.9{\AA}]   &
                               \ife{5J88}{1}{DB} (3.32{\AA}) [1.22{\AA}] \\
    NR\_all\_39327.1 &  3.8  & \ife{4V6M}{1}{AV} (7.1{\AA}) [3.3{\AA}]   &     
                               \ife{4V6M}{1}{AV} (7.1{\AA}) [3.3{\AA}] &
                               \ife{4V6M}{1}{AV} (7.1{\AA}) [3.3{\AA}] \\
    NR\_all\_39428.1 &  2.6  & \ife{4V8U}{1}{CV} (3.7{\AA}) [1.1{\AA}] &
                               \ife{4V8U}{1}{CV} (3.7{\AA}) [1.1{\AA}] &
                               \ife{4V8U}{1}{CV} (3.7{\AA}) [1.1{\AA}] \\
    NR\_all\_44399.2 &  2.96 & \ife{4V70}{1}{A1} (17{\AA}) [14.04{\AA}] \\
    NR\_all\_48374.2 &  2.1  & \ife{1C04}{1}{F} (5{\AA}) [2.9{\AA}] &
                               \ife{1C04}{1}{F} (5{\AA}) [2.9{\AA}] &
                               \ife{1C04}{1}{F} (5{\AA}) [2.9{\AA}] \\
    NR\_all\_55323.1 &  3.71 & \ife{4V68}{1}{AY} (6.4{\AA}) [2.69{\AA}] \\
    NR\_all\_59913.2 &  2.1  & \ife{4V4Q}{1}{CA} (3.46{\AA}) [1.36{\AA}] &
                               \ife{4V4Q}{1}{CA} (3.46{\AA}) [1.36{\AA}] &
                               \ife{4V4Q}{1}{CA} (3.46{\AA}) [1.36{\AA}] \\
    NR\_all\_62116.2 &  2.1  & \ife{4V54}{1}{DB} (3.3{\AA}) [1.2{\AA}] \\
    NR\_all\_68375.1 &  8.3  & \ife{3IZ4}{1}{A} (13.6{\AA}) [5.3{\AA}] &
                               \ife{3IZ4}{1}{A} (13.6{\AA}) [5.3{\AA}] &
                               \ife{3IZ4}{1}{A} (13.6{\AA}) [5.3{\AA}] \\
    NR\_all\_87281.1 &  3.9  & \ife{3J3V}{1}{B} (13.3{\AA}) [9.4{\AA}] \\
    NR\_all\_95973.1 &  3.8  & \ife{4D61}{1}{j} (9{\AA}) [5.2{\AA}] &
                               \ife{4D61}{1}{j} (9{\AA}) [5.2{\AA}] &
                               \ife{4D61}{1}{j} (9{\AA}) [5.2{\AA}] \\
    NR\_all\_97012.1 &  2.0  & \ife{4V49}{1}{AW} (8.7{\AA}) [6.7{\AA}] &
                               \ife{4V49}{1}{AW} (8.7{\AA}) [6.7{\AA}] &
                               \ife{4V49}{1}{AW} (8.7{\AA}) [6.7{\AA}] \\
    \bottomrule
  \end{tabular}
  \caption{}
  \label{tab:rep-res-diff-details}
\end{table}

The new method will not select a cryo-EM structure if an X-ray structure is
available, however, new cryo-EM structure are being solved with increasingly
higher accuracy, rapidly approaching the quality of X-ray structures. This is a
potential reason for the differences found in Table~\ref{tab-res-diff-details}.
I examined this by determining the difference between the resolution of the
representative structure and the minimal resolution in that group for all
groups, using only X-ray structures and repeating the analysis. Shown in
Figure~\ref{fig:xray-only-diff} is this figure.

\begin{figure}
  \caption{}
  \label{fig:xray-only-diff}
\end{figure}

In this figure we can see that there are very few groups have resolution
difference greater than 1\AA{} for all methods. As with the previous analysis, all
methods perform similarly in terms of the number of groups with large
differences. However, the Dual 1\% method does show a smaller resolution
difference. Shown in Table~\ref{tab:xray-only-outliers} are the groups with large
difference between the representative resolution and minimum resolution.

\begin{table}
  \begin{tabular}{llrrr}
    \toprule
    Group &  Minimum Resolution ({\AA}) &  Previous &  Dual Increase  & Dual 1\% Increase \\
    \midrule
    \bottomrule
  \end{tabular}
  \caption{}
  \label{tab:xray-only-outliers}
\end{table}

From this table we can see that the Dual 1\% increase has the fewest differences.
In addition, the times it does have issues are the same as the previous method,
indicating this method is not making new mistakes.

Overall, I note that the new method, Dual 1\% Increase, almost always, greater
than 90\% of the time, selects a representative that has a resolution close to
the best nominal resolution in the group. This is desirable because higher
resolution generally means a better structure. However, there are very few cases
w’here this does not occur. The new method performs better than the previous
version as there are fewer of these cases and the difference in resolution is
smaller. Many of these cases appear to be due to the group containing both
cryo-em and x-ray structures. The new method, by construction, will always
prefer X-ray structures, even when the nominally higher resolution is a cryo-EM
structure.

\subsubsection{Independent measures of representative selection}

PDB provides quality reports for all structures \cite{Gore2012}. These quality
reports use a variety of metrics that were suggested by the X-ray validation
task force \cite{Gore2012}. Here I evaluate our selection on the basis of data
from these quality reports. Shown in Figure~\ref{fig:ec-lsu-report} is the
quality report for the representative selected by the dual 1\% increase method
as compared to that of the previous method. From this figure we can see that the
dual 1\% increase method selects structures with fewer RSRZ outliers and similar
RNA backbone scores, however, all other metrics are worse.

\begin{figure}
  \caption{Screenshots of the summaries of quality reports for 4V54 (left), the
    representative selected by the Dual 1\% increase method as compared to the
    report for 5J91 (right), the representative selected by the previous
  method.}
  \label{fig:ec-lsu-report}
\end{figure}

The same data is shown for the Thermus thermophilus large subunit group in
Figure~\ref{tt-lsu-report}. In this case we see that both structures have very
similar metrics overall. The previous method does select a structure, 5HCQ, with
fewer RSRZ outliers.

\begin{figure}
  \caption{Screenshots of the summaries of quality reports for 5FDU (left), the
    representative selected by the dual 1% increase method as compared to the
    report for 5HCQ (right), the representative selected by the previous
  method.}
  \label{fig:tt-lsu-report}
\end{figure}

These figures indicates that while the dual 1\% increase method does not use the
quality metrics, it does select a good structure.

\subsection{Stability of the Representative Set}

In order to asses the stability of our method we computed the representatives
for a set of 91 weekly releases from 2.0 to 2.90 covering releases from December
5, 2014 to August 26, 2016 using the three methods. We then plotted the number
of changes in the representatives for all equivalence class that are common to
all releases. We expect that the any increase method will be the most unstable,
with the previous and current method being comparable. Our goal is the current
1\% increase method will be more stable. The summary is shown in
Figure~\ref{fig:rep-changes}. This plot summarizes the total number of changes
for 1449 classes that exist in all 91 releases. We can see that all three
methods are comparable in stability, as they all end near 25 total changes. In
general both the 1\% increase and the previous method are more stable, show fewer
changes, than the any increase method. For most of the plot our current method
shows as many or fewer changes than the previous, indicating it is slightly more
stable.

\begin{figure}
  \caption{A cumulative distribution of the total number of changes for all
    three methods. This plot is a cumulative sum of the total number changes for
    all classes at the ``all'' resolution cutoff which exist in all of the 90
  releases used in the analysis.}
  \label{fig:rep-changes}
\end{figure}

We examined the number of total number of changes for each class in the dataset.
As shown in Table~\ref{tab:rep-changes-count} most classes that exist in all the
releases show no changes. Of those that show changes the majority change only 1
time.

\begin{table}
  \begin{tabular}{lrrrrr}
    \toprule
                        & \multicolumn{5}{c}{Number of Changes} \\
    % \cmidrule{r}{2-6}
    Method              & 0    & 1  &  2 &  3 &  4 \\
    \midrule
    Previous Method     & 1528 & 19 &  0 &  2 &  0 \\
    Any Increase Method & 1530 & 17 &  0 &  3 &  0 \\
    1\% Increase Method & 1529 & 16 &  0 &  2 &  1 \\
    \bottomrule
  \end{tabular}
  \caption{Number of changes for classes that exist from 2.0 to 2.90. This table
    summarize the number of classes that show 0, 1, 2, 3, or 4 changes amongst
  the classes that exist in all releases. }
  \label{tab:rep-changes-count}
\end{table}

To see how often classes with more than 1 change alters the representative we
plotted all classes that show more than one change across these releases as seen
in Figure~\ref{fig:multi-change}. We see that there are only 3 classes that show
this behavior: 80570, a 2 nucleotide synthetic RNA, 83717 the \EC{} LSU, and
97519 the T. t LSU. From this we can see that all methods are generally stable
for many releases followed by a few very quick changes. This is more
desirable over a method which changes very frequently. One class, 80570
shown in red, the representative is changed every time a new structure is
added by all methods. This class has few members, 8 in 2.90, and a very
few nucleotides, leading to minor changes in resolution changing the
selected representative. Shown in Figure CONSTANT CHANGE are the
structures of the representative in 2.0 and 2.90. From this we can see
that the structures are highly similar

\begin{figure}
  \caption{This figure shows the rate of changes in for each class that changes
    more than once between 2.0 and 2.90. Each colored line represents a class,
    and each mark indicates when members are added to the class. The three sets
  panels indicate the three different methods tested here.}
  \label{fig:multi-change}
\end{figure}

\section{Conclusions and future work}

In summary, we have developed a method to select a representative structure from
our equivalence classes. This method reliably selects a good structure from the
set of structures. It is also stable and will not change due to small
improvements in the structure.

One possible improvement is to use the quality metrics provided by PDB, such as
real space refinement (RSR) and the RSR Z-score (RSRZ) in the selection of the
representative structure. These are measures of how well a modeled structure
fits the underlying data from the experiment. This data and its utility will be
discussed in detail in the following chapter. Usage of this data will allow us
to detect if a structure is well modeled. This could be used in place of our
base pairs per nucleotides metric to select structures. The downside is that not
all structures have RSRZ data. Notably, cyro-em structures do not yet have such
a metric. The community is working on such metrics however.
