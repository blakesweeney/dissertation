\chapter{SELECTION OF LOOPS FROM THE REPRESENTATIVE SET}

In this chapter I will discuss the changes I have made to the way the BGSU RNA
pipeline selects loops for building the RNA 3D Motif Atlas
(\url{http://rna.bgsu.edu/rna3dhub/motifs}). The goal of our Motif Atlas is to
maintain a high quality, up-to-date clustering of representative structured from
the 3D structure database on an ongoing basis. An important step in of building
the motif atlas is the selection of loops. The original process of selecting
loops was implemented by Dr. Anton Petrov in previous work \ref{Petrov2012}.
Overall, the new procedure works as follows: First, all loops from all
representative IFE’s from the 4.0Å equivalence classes are extracted. Then each
loop is examined for the presence of possible structural flaws, which serve as
criteria for removal. The criteria are: 

\begin{enumerate}
  \item Presence of chain breaks in the loop. Chain breaks occur when the
    phosphate backbone of the molecule is not continuous. This occurs due to
    modeling issues, or a lack of electron density in the affected region.

  \item Presence of a self-complementary sequence in a nominally unpaired
    internal loop. These loops are often a poorly modeled helixes that appear as
    loops in our method.

  \item Presence of incomplete nucleotides. Incomplete nucleotides occur when
    only part of a nucleotide is modeled. In some structures only electron
    density for the backbone is observed, leading to models that lack the base
    of a nucleotide in some positions
\end{enumerate}

Examples of these structural issues are shown in Figure STRUCTURAL ISSUES. These
criteria all examine the structure of the loops for signs of data incompleteness
or modeling flaws. If any loop shows one these problems it is rejected from
clustering. 

In addition to these structural issues, the previous also rejects loops that
have been incorrectly extracted. This occurs when a single loop contains too
many chains.

Finally, the previous method rejects loops that contain modifications. This is
because our current FR3D cannot process modified bases \cite{Sarver2008a}. We are
currently working on removing this limitation.

My changes to the selection process are in three areas. First I improved the
selection criteria to take advantage of mmCIF data. For example, mmCIF data
includes information on occupancy. Secondly, I altered the logic to exclude
loops from cyro-EM data. Finally, I have used a measure of modeling quality,
Real Space R Z-Score (RSRZ), to exclude loops. I will discuss each of these
changes in turn.

\section{Modifications to Previous Loop Selection Methodology}

The first change was to take advantage of mmCIF data. This change affected two
of the criteria, how the presence of chain breaks is detected, and how the
presence of incomplete nucleotides is detected. 

Previously, chain breaks were detected by measuring the distance between the
phosphate atom of one nucleotide and the O3’ of the preceding nucleotide. This
distance should be less than 2Å (CITE ANTON DISS). This is a geometric cut off
with a free parameter that requires selection. The new format, mmCIF, provides a
mapping between the observed sequence and the experimental sequence. In this
mapping, chain breaks are detected as nucleotide  positions in the experimental
sequence that are not mapped to observed sequences. By using the experimental
sequence mapping we no longer need to measure distances in the structure. 

The second affected criteria, presence of incomplete nucleotides has been
extended to include atoms with zero occupancy. The previous procedure rejected a
loop if it contained any nucleotide that lacked all the required atoms. This
occurs when an author chooses not to model a nucleotide because of missing data.
However, some authors will model nucleotides when there is missing data and then
mark one or more of the atoms as having zero occupancy (CITE). The procedure
will now detect these cases by checking the occupancy data for each atoms of the
nucleotide. In the case that one or more atoms in a loop has zero occupancy, the
loop is excluded from clustering. The new logic is more rigorous and removes
more loops than the previous method. Table CHANGES compares the new selection
criteria to the previous one.

Finally, when beginning mmCIF data I found a bug in our loop extraction
procedure that inadvertently extracts loops formed by molecules related by
symmetry operators in the crystal structures. While this bug has been fixed I
have added a new criteria to ensure that we do not extract loops with more than
one symmetry operator. 

These changes alter the loop selection logic. Shown in Table CHANGES is a
comparison between the new and previous method. I took all loops from the
previous motif atlas release that were clustered into motifs which still exist
in the current data set and determined their status using the new logic. I then
summarized it as counts. From this we can see that the majority of loops will
have the same classifications, but a small percent do change.

\begin{table}
  \begin{tabular}
  \end{tabular}
  \caption{A table showing the numbers of loops affected by the changes to the
    selection procedure to address incomplete or flawed loop data or models.
  This uses loops from motif release 1.18 compared to NR release 2.92}
  \label{tab:loop-quality-changes}
\end{table}

\section{Rejection of loops from Cyro-Electron Microscopy}

The next change to loop selection, rejection of loops from cyro-EM structures,
is a simple addition with a large effect. Currently, X-Ray structures are the
‘gold standard’ of structural methods (CITE). There is a great deal of exciting
work in cyro-electron microscopy (cyro-em) (CITES) however this technique is
under rapid development and lacks defined criteria for assessing structural
quality studies and standards that are already well-established for X-ray
structures. The loop extraction procedure will no longer use loops that come
from structures solved by cyro-EM. 

Rejecting loops from cyro-em structures has a large effect on the number of
loops for available clustering. As a test of this I extracted all loops from the
representative set described in the previous chapter and then counted the number
of valid loops from each source.  Shown in Table SOURCE COUNTS is a summary of
counts of valid internal loops. 

\begin{table}
  \begin{tabular}
  \end{tabular}
\end{table}

The comparison in table SOURCE COUNTS shows that there is a much larger growth
in number of loops extracted from cryo-EM structures over the past DD months. 

By not using cyro-EM loops, we reduce the number loops available for clustering
greatly. Notably it reduces the number of loops in release 2.85 by half bringing
the number of selected internal loops, 1945, to fewer than number used in the
previous motif release, 2410. This change has a large effect on the selection of
loops and thus the final clustering of motifs. 

\section{Usage of Real Space R data for loop selection}

The final change to the loop selection procedure is to use modeling quality data
as an addition criterion measuring how well 3D models reflect the underlying
electron density. Our previous selection procedure successfully detected
structural deficiencies of the loops but did not consider modeling issues
(CITE). The new changes uses the recently made available by PDB for each X-ray
structure Real Space Refinement (RSR) and RSR Z-Score (RSRZ) data. RSR directly
measures how well the 3D model fits the underlying experimental electron
density. I begin with a discussion of the new RSRZ data and the discuss the
modifications to our quality criteria.

\subsection{Description of Real Space Z-Score}

Real Space Refinement (RSR) is a measure of how well the 3D model present in a
PDB file fits the observed, underlying electron density obtained by X-ray
diffraction. There are several issues with this measure however (CITE
Statistical quality indicators for electron-density maps). To deal with these
issues a new statistical measure has been proposed (CITE Xray validation), RSR
Z-Score. PDB provides a normalized version of RSR, called Real Space Z-Score
(RSRZ). This data is calculated for each structure on a residue level. The data
is produced by computing a Z-Score for each type of residue present in the
structure and comparing to like residues in all other comparable structures, ie
A’s are compared to other A’s, while C’s are compared to other C’s, of similar
resolution (CITE). PDB provides this data as part of their validation reports
and the data is imported as part of our weekly update pipeline. In this way the
quality of each structure is compared to other structures of comparable
resolution, as well as all other X-ray structures in PDB.

\subsection{Determining how to use RSRZ to reject loops}

We begin by examining the distribution of RSRZ data for each type of loop in the
representative set of structures created in the previous chapter. This release
corresponds to the 2.85 representative structure release avaiable at
(http://rna.bgsu.edu/rna3dhub/nrlist/release/2.85). The distribution of such
data is shown in Figure RSR Distribution. In this figure we can see that loops
nucleotides have a median value nucleotides in loops have an RSRZ value of near
0, indicating a good fit. We can also see a long tail in the RSRZ distribution
that indicates a large range in the quality of structures. PDB assigns an RSRZ
value of 2 to indicate a large outlier. This is shown on the graphs as a red
bar.

\begin{figure}
\end{figure}

From this distribution we can see that median value is less than 0. This
indicates that loops in bases from our representative structures are better
modeled than the average nucleotide. This supports the conclusions of the
previous chapter that appropriate criteria are used to select the representative
structure. However, there is a long tail of nucleotides with very high values.
Loops with containing poorly modeled nucleotides that should be excluded from
clustering and inclusion in the 3D Motif Atlas. I used RSRZ values to create two
criteria. The first is that all bases in a loop must be well modeled, that is
all nucleotides in the loop should have a low RSRZ value. I use this cutoff
because we want to ensure that a loop is not an ‘appealing fiction’ where all
bases in it are poorly modeled. We do not consider if the bases are interacting
in this criterion because some loops do not contain interactions. Secondly, is
that all nucleotides that make interactions with other bases within the loop
must have a low RSRZ value as well. This criterion is used because we expect
that loops interacting within a loop will have a lower RSRZ value by virtue of
being ‘fixed in place’ in the crystal.

\subsection{Detecting poorly modeled loops}

The first criterion is that all bases in a loop should be sufficiently well
modeled. A well modeled base has a low RSRZ value, thus a poorly modeled loop is
one where all modeled bases in the loop have a high RSRZ value. I will refer to
these loops as ‘poorly modeled’. We are interested in explore the effect of
various RSRZ cutoffs on motif clustering so we will use a range of values. We
have been advised by PDB to implemented an RSRZ threshold of 2 to identify loops
that are poorly modeled. I tested cutoffs of 1, 1.5 and 2.5 to explore the
effect of various cutoffs on motif clustering. I applied this criterion, all
nucleotides in the loop must have an RSRZ less than the cutoff, to all valid
loops in the dataset. A summary of the number of loops sorted by this criterion
into successive ranges of RSRZ in Table CUTOFFS REJECT SUMMARY. 

\begin{table}
  \begin{tabular}
  \end{tabular}
  \caption{Number of loops Exceeding different RSRZ cutoff values: Counts of the
    number of loops of each type rejected by the criteria that requires all
    nucleotides in a base to pass the RSRZ cutoff. The percentages are the
    percent of loops rejected out of all loops. The data is for all valid X-ray
  loops from the NR release 2.85.}
\end{table}


As expected, the lower the values of the RSRZ criterion, the more loops are
rejected. However, even for the lowest value used, RSRZ value of 1, relatively
small small number of loop instances are rejected . This is suggests that our
current set is not heavily populated with loops that do not model the data well.

I evaluated how well structured the rejected loops are by determining the number
of interactions per nucleotide for all

Inspection of the rejected loops shows that some appear to be poorly structured
such as EXAMPLE. 
 
Others loops, such as HL\_2OIU\_002, appear to be well structured loops but poorly
modeled. We believe this is because the modeling procedure for a new structure
is to begin with a previous structures and modify parts to better fit the data.
In some regions the old is not modified enough to fit the data, or new
experiment has no density in this region. This leads to the creation of an
‘appealing fiction’, where the structure appears reasonable but it is not
experimental supported.

\subsection{Detection of loops containing poorly modeled interacting nucleotides}

The next use for RSRZ is to explore if the relationship between RSRZ values and
the number of of annotated interactions. We expect that nucleotides that are
making more interactions will have a lower RSRZ score than nucleotides that make
fewer interactions. This is because the lack of interactions may allow the base
to adopt a variety of conformations in the crystal, leading to diffuse electron
density. With diffuse data any single modeled position fits experimental
electron density poorly resulting in a high RSRZ value. 

I explored this by plotting the distribution of RSRZ values for all nucleotides
in loops vs the number of FR3D annotated interactions the BGSU database records
for each nucleotide as shown in Figure NT RSRZ DIST. This figure is a ‘violin
plot’, which were originally developed in (CITE Violin plots : A box
plot-density trace synergism). The figure is a mix of a density estimation and a
box plot. In this figure all nucleotides making the same number of interactions
are grouped together and a density estimate for each group is produced. The
estimate is plotted along each row and is referred to as a ‘density trace’ or
‘trace’ for this description. The range of the trace indicates the range of the
values. The width of the trace at any region indicates what fraction of all
nucleotides are in this region.

In this figure I included all types of annotated interactions, base pairs, base
stacks, and base backbone interactions in the counts. This plot shows that there
is a broad range of RSRZ values for regardless of the numbers of interactions.
This range is largest for nucleotides with 0 interactions.  Secondly. the
highest median value occurs with bases that make no interactions, and this
gently decrease as the number of interactions increase. The RSRZ value for
nucleotides in internal loops with zero interactions is NN +- NN, while for 3 it
is NN +- NN (Table RSRZ MEANS)

\begin{figure}
  \caption{ Distribution of RSRZ values for nucleotides in loops clustered by
    the total number of annotated interactions they form with other nucleotides
    in the loop. The nucleotides are grouped by the total number of interactions
    formed, including basepairs, base backbone, and base-stacking interactions.
    The distributions of the RSRZ values are represented as violin plots. The
    blue dots and inner bars within each ‘bubble’ represent the mean and
    standard deviation of the data respectively. The data is plotted vertically
    to better display the trend of decreasing mean RSRZ as the number of
  interactions for each nucleotide increases. }
\end{figure}

The large range of RSRZ makes it difficult to examine the number of outliers in
each group. To alleviate this I created a truncated version of this plot to
visualize the number of outliers for each number of interactions. In this figure
the RSRZ values are truncated at a value of 3. Thus all nucleotides with an RSRZ
value above 3 was set to 3. 

\begin{figure}
  \caption{Violin plot showing truncated distributions of RSRZ values for
    nucleotides in loops clustered by the total number annotated interactions
    the nucleotide makes. This plot is the same data as Figure NT RSRZ DIST but
    the displayed RSRZ has been clamped to a maximum value of 3. Outlier
    instances (RSRZ > 3) are represented at RSRZ = 3. Numbers above each violin
    indicate the number of nucleotides with the given number of interactions.
    The means and standard deviations are still computed with the untruncated
    data however. The data is plotted horizontally to better compare the shapes
  of the distributions.}
\end{figure}

From this more focused figure we can see that there are far fewer RSRZ outliers
for Nts forming three or more annotated interactions (the tails are skinnier as
interactions increase) across all loop types. 

These observations generally confirm the expectation that nucleotides the
interactions have lower mean RSRZ values and fewer outliers. Thus I implemented
an additional criterion, that all interacting bases in a loop must have RSRZ
values below the cutoff for the loop to be included. As with the previous
criterion it is necessary to determine its effect on clustering the effect on
clustering. Shown in Table INTERACTION BAD COUNTS is a summary of the number of
loops excluded by this method for each cutoff value. In other words, a loop is
excluded if one or more interacting nucleotides has RSRZ > the threshold.

\begin{table}
  \begin{tabular}
  \end{tabular}
  \caption{Counts and percent of all loops extracted that are rejected by
    requiring all bases with annotated interactions passing each RSRZ cutoff.
    The percentages are the percent of all loops that are rejected by the
  cutoff.}
\end{table}

As before increasing the RSRZ cutoff decreases the number of loops excluded.
Here we can see a much larger effect of this criterion. The highest number of
loops excluded, 400, occurs for internal loops with a constraint of RSRZ > 1, on
the other hand, the largest number of loops rejected for the previous constraint
is 45 for hairpin loops at RSRZ > 1. 

We use the two criteria described here to build a new RSRZ quality assurance
step for loops; all nucleotides in a loop must have an RSRZ value below the
first cutoff value and all interacting bases must have the RSRZ below the second
cutoff value to be accepted. I use a range of cutoffs, 1, 1.5, 2, and 2.5 to
explore the effect on number of loops excluded. For the study here I used the
same RSRZ value for both cutoffs. The total number of loops rejected by this
method are summarized in Table RSRZ REJECT SUMMARY.

\begin{table}
  \begin{tabular}
  \end{tabular}
  \caption{Counts and percent of all loops extracted that are rejected by the
    RSRZ tested here. The percentages are the percent of all loops that are
  rejected by the cutoff.}
\end{table}

\section{Exploring the effects of the new quality criteria on motif clustering}

To explore the effect of the new quality metrics on clustering, I built a Motif
Atlas release of internal loops using the representative release built in the
previous chapter. This clustering was used to test the effects of the effects of
new quality criteria on motif clustering. The quality criteria did not include
the RSRZ cutoffs developed above. By doing this we are able to build a motif
atlas and then examine what would happen if the loops were rejected by RSRZ
cutoffs. Doing so makes the comparisons of the resulting motif clusterings
simpler because changing the loops used will have many effects on the resulting
clustering.

I discuss the effects of the new quality criteria on the loops, followed by a
general description of the motif clustering and then discuss the effects of
applying the RSRZ cutoffs to the clustering. 


\subsection{Summary of the new loop quality criteria}

In the first step in the motif clustering is to select loops that pass each
quality check. Shown in Table CURRENT QUALITY are the numbers of loops from the
representative IFE’s from the 4.0Å resolution cutoff rejected by each criterion.

\begin{table}
  \begin{tabular}
  \end{tabular}
  \caption{Counts of the loops that are rejected by each quality criteria for
  all loops that come from a representative IFE from the 2.85 representative
release with resolution cutoff of 4.0Å.}
\end{table}


From this table we can see that 151, or 3\%, internal loops are rejected on the
basis of modified nucleotides. This differs from the previous clustering where
0.9\% of loops were rejected (CITE). In addition, the largest fraction of loops,
45\%, that are rejected are due to being cyro-EM loops

\subsection{Description of the properties of the new motif atlas}

I used the loops that passed our updated quality checks, but not the new RSRZ
constraints, that were less than 15 nucleotides in length to build a motif
atlas. The additional constraint of 15 nucleotides only reduced the total number
of valid loops to 1757 from 1945 and was done to ensure the analysis completed
quickly. This clustering produced a grouping with 256 motif families. This is a
large change from the 372 motif families in the previous release. Shown in Table
SUMMARY OF COUNTS is a summary of the number of motif families from the release
described here as compared to the previous release. The table includes the
counts of singleton motif families, families with only one instance, and the
recurrent motifs, families with more than one instance.

\begin{table}
  \begin{tabular}
  \end{tabular}
  \caption{ A table summarizing the number of loops in the previous motif
  release as compared to the new release. Percentages are out of the total
number of motifs in that release.}
\end{table}

From this table we can see that the overall ratio of singleton to recurrent
groups remains relatively constant in both releases, with both clusterings being
nearly evenly split between singleton and recurrent motifs. In addition we see
that a large fraction of the previous singletons, 29% of all motif groups
overall and 62% of all singleton groups, come from cryo-em structures. There is
a smaller fraction of singletons now, 43% as compared to the previous 47%,
however, the change is fairly small.

I next examined the sizes of the motifs in the new clustering. Shown in Figure
NUM INSTANCES is the size distribution in terms of number of motif instances in
all motifs motif. From this figure we can see there are 50 singleton motif
groups and two groups with around 250 members. 

\begin{figure}
\end{figure}

This figure shows a large spike at 1, indicating the large number of singleton
motifs in the clustering. The majority of motifs have fewer than 50 instances,
with only 2 motifs having near 250 instances. 

The next consideration is the distribution of the number of nucleotides in the
motifs. Show in Figure NUM MOTIF NTS is the distribution of the number of
nucleotides in each motif. 

This figure shows the overall size distribution of internal loop motifs for all
motifs in the new motif release. It is useful to know the sizes of each class of
motifs, singleton and recurrent. Shown in Figure NUM NT BY CLASS is this
information. This figure shows that singleton motifs tend to be larger, with a
mean of 9.5 as opposed to the mean of 8.5 for recurrent motifs. 

\subsection{Exploration of the effects of RSRZ cutoffs on the motif clustering}

I then explored what would happen to the motif atlas when the RSRZ cutoffs
applied. This was done by examining each motif for loops that are rejected by
each RSRZ cutoff and seeing the motif would still exist. I explored two main
questions. First, what type of motifs are changed? Secondly, are the loops I am
removing geometrically distinct from all other loops? 

To answer the first question I explored the distributions of the number of
instances group as well as the number of nucleotides in each motif  that
contains at least one outlier. Shown in Table NUMBER MOTIFS ALTERED is the
summary of the number of motifs altered at each cutoff. 

\begin{table}
  \caption{A table showing the number of motifs for singletons vs recurrent
    motifs with rejected loops for each cutoff tested here. The counts are the
    number of loops rejected by each cutoff while the percents in the
    parenthesis are the fraction of all motifs of that type that are are
    affected by the cutoff. The upper left column indicates that there are 120
    total motifs that contain rejected loops, and this is 46% (120/256) of all
    motifs, while the column to the right indicates that 33% (36/109) of all
  singleton motifs are rejected by the RSRZ > 1 cutoff.}
\end{table}

From this table it appears that the majority of motifs affected are recurrent
motifs, 57% of recurrent motifs contain at least one loop that is rejected as
compared to 33% percent of the singletons that contain a rejected loop. However,
this could be due to the fact that there are many fewer loops that appear in a
singleton family, 109, as compared to recurrent motifs, 1685. To test if the
RSRZ criteria effects singleton groups more than recurrent motifs I created a
table that shows the counts of loops in each class, singleton vs recurrent
motifs that are rejected as normalized by the total number of loops shown in
Table LOOP IN MOTIF FRACTION. This table shows that vast majority of rejected
loops, greater than 85% for all cutoffs, occur in recurrent motifs. From this I
infer that the singleton motifs are generally modeled correctly. However, the
recurrent motifs  often contain poorly  modeled loops. This is consistent with
the hypothesis that poorly modeled loops occur because a structure was used as a
template and a loop was not modified to fit the new data well.  

\begin{table}
  \caption{A table showing the counts of rejected loops in each type of motif,
    singleton or recurrent. The percents in the parenthesis indicate the percent
    of rejected loops that occur in each type of motif. Thus the upper row shows
    that there are 419 loops rejected using the RSRZ > 1 cutoff, of those 36 or
    9%, 36/419, occur in singleton motifs while 383 or 91%, 383/419, occur in
  recurrent motifs. }
\end{table}

To answer the second question I created a histogram of the minimum for excluded
loops to all other loops it’s motif group show in Figure EXCLUDE MIN DISC HIST.
From this figure we can see that the RSRZ cutoffs appear to reject outliers
within groups as the distribution of discrepancy values peaks near 0, indicating
that there is always at least one other loop that the outlier is similar to. 

\begin{figure}
  \caption{A histogram of the minimum discrepancy from a rejected loop to all
    other loops in the motif family. Lower values indicate a more similar pair
    of instances. Our motif atlas requires that all instances in the same motif
  family have discrepancy less than 1Å/nt.}
\end{figure}

This figure shows that the excluded loops are often similar to at least one
other member of their family. This indicates that our RSRZ criteria is not
removing geometric outliers from each motif family. 

The RSRZ criteria appear to primarily exclude members of recurrent motifs that
are geometrically similar to other members of the motif family. RSRZ criteria
have a much smaller effect than than other motifs, most notably the criteria to
not use any loops from cyro-em structures. 


\section{Conclusions and future work}

Here I have described improvements to the previous loop selection criteria, as
well as a new method that uses a measure of modeling quality, RSRZ, to better
select loops. After investigating the effects of these new criteria on
clustering we see that the largest effect comes from rejecting loops from
cyro-em structures. Usage of RSRZ does not appear to improve the homogeneity of
groups, as seen by examining the discrepancies of rejected loops. However, 

Future work may focus on exploring other measures such as number of clashes in a
motif. In addition, it is possible to calculate RSRZ by parts of nucleotides,
base separate from backbone. Doing so could provide a more refined measure of
the modeling quality.

